\chapter{Group Theory}
\section{Brief Discussion on Group of Units modulo N}\label{unit}
We will discuss a bit about the group of units. Let  $ N= 2^k p_1^{k_1}\cdots p_n^{k_n} $ where $ p_i $s are odd primes. By CRT, we have $ (\Z_N)= (\Z_{2^k})\times (\Z_{p_1^{k_1}}) \times \cdots\times (\Z_{p_n^{k_n}}) $. We have the unit group $$ (\Z_N)^\times= (\Z_{2^k})^\times\times (\Z_{p_1^{k_1}})^\times \times \cdots\times (\Z_{p_n^{k_n}})^\times $$. For odd prime powers, we have that the unit group is cyclic $ (\Z_{p^{k}})^\times = \Z_{p^k-p^{k-1}} $.\\
For $ 2^k $ we have $ (\Z_{2})^\times = \Z_1 $ the trivial group, $ (\Z_{4})^\times = \Z_2 $ cyclic and $ (\Z_{2^k})^\times = \Z_2 \times \Z_{2^{k-2}} $ noncyclic groups for $ 2^k\ge 8 $. So the only time where the unit group is cyclic is $ N=1,2,4,p^k, 2p^k $ where $ p $ is an odd prime.\\
\section{Solution}
\textbf{G1}: Let $ H $ be a normal subgroup of a group $ G $, and let $ K $ be a subgroup of $ H $.
\begin{itemize}
	\item[(a)] Give an example of this situation where $ K $ is not a normal subgroup of $ G $,
	\item[(b)] Prove that if the normal subgroup $ H $ is cyclic, then $ K $ is normal in $ G $. 
\end{itemize}
\soln
\begin{itemize}
	\item[(a)] Let $ G=S_4 $, $ H=A_4 $, and $ K= \{e,(123),(132)\} $.
	\item[(b)] Let $ H = <h>$ be cyclic. Let $ K= <k> $ where $ k=h^a $ for some $ a\in \mathbb{N} $.\\
	Since $ H $ is normal, $ ghg^{-1}=h^b \in H$ for some $ b $.\\
	$ gkg^{-1}= gh^a g^{-1}= (ghg^{-1})^a= h^{ba}=k^b\in K.$ So, $ K $ is normal in $ G $.
\end{itemize}
\qed\\
\textbf{G2:} Prove that every finite group of order at least three has a nontrivial automorphism. 
\soln We will try this in two cases:\\
Case 1: The group is not abelian. Let $ g\notin Z(G) $. Let $\phi_g $ be the nontrivial automorphism $ h\mapsto ghg^{-1} $.\\
Case 2: The group is abelian. If there is an element of order not equal to 2, the inverse map is a nontrivial automorphism. If every element is of order 2: $ G= (\mathbb{Z}/2\mathbb{Z})^n $, where $ n>1 $. Swap 2 elements.\qed\\
\textbf{G3:}
\begin{itemize}
	\item[(a)] State the structure theorem for finitely generated Abelian group.
	\item[(b)] If $ p $ and $ q $ are distinct primes, determine the number of nonisomorphic Abelian groups of order $ p^3q^4 $.
\end{itemize}
\soln
\begin{itemize}
	\item[(a)] If $ G $ is finitely generated Abelian group, $ G $ is isomorphic to $ \mathbb{Z}^n \times \mathbb{Z}_{a_1}\times \cdots \times \mathbb{Z}_{a_r}$ where $ a_i\mid a_{i+1} $, $ \mathbb{Z}_a = \mathbb{Z}/a\mathbb{Z}$ cyclic group of order $ a $. 
	\item[(b)] Let $ P(n) $ be the partition function. The number of nonisomorphic Abelian groups of order $ p^3q^4 = P(3)P(4)=3\times 5 = 15$.
\end{itemize}
\qed\\
\textbf{G4:}Let $G=\mathrm{GL}\left(2, \mathbb{F}_p\right)$ be the group of invertible $2 \times 2$ matrices with entries in the finite field $\mathbb{F}_p$, where $p$ is a prime.
\begin{itemize}
	\item[(a)] Show that $G$ has order $\left(p^2-1\right)\left(p^2-p\right)$.
	\item[(b)] Show that for $p=2$ the group $G$ is isomorphic to the symmetric group $S_3$.
\end{itemize}
\soln Let $G=\mathrm{GL}\left(2, \mathbb{F}_p\right)$.\\
\begin{itemize}
	\item[(a)] Choosing a invertible $ 2\times2 $ matrix is equivalent to choosing two linearly independent vectors(which will be the columns of the matrix) from the space $ \mathbb{F}_p^2 $. We can choose a nonzero vector in $\mid \mathbb{F}_p^2\mid -1 =  p^2-1 $ ways and the second vector can't be a multiple of the first vector(there are $ p $ of them). So, we can choose the second vector in $ p^2-p $ ways.
	\item[(b)] The group is of order $ 6 $. We just have to show that it is not abelian. Show for the elements $a= \matr{0}{1}{1}{0} $ and $b= \matr{1}{1}{0}{1} $. $ ab=\matr{0}{1}{1}{1}, ba= \matr{1}{1}{1}{0}. $
\end{itemize}
\qed\\
\textbf{G5:} Let $G$ be the group of units of the ring $\mathbb{Z} / 247 \mathbb{Z}$.
\begin{itemize}
	\item[(a)] Determine the order of $G$ (note that $247=13 \cdot 19$ ).
	\item[(b)] Determine the structure of $G$ (as in the classification theorem for finitely generated abelian groups). Hint: Use the Chinese Remainder Theorem.
\end{itemize}
\soln See Section \ref{unit}.\\


So, for $ N=247 $ the order of the group is $ 12\times 18=216 $. And the structure of $ G $ is $ \Z_{12} \times \Z_{18}= \Z_{3}\times \Z_4 \times \Z_9 \times \Z_2 = \Z_{6}\times \Z_{36}$.\\
\qed\\
\textbf{G6:} Let $G$ be the group of invertible $2 \times 2$ upper triangular matrices with entries in $\mathbb{R}$. Let $D \subseteq G$ be the subgroup of invertible diagonal matrices and let $U \subseteq G$ be the subgroup of matrices of the form $\left[\begin{array}{ll}1 & x \\ 0 & 1\end{array}\right]$ where $x \in \mathbb{R}$ is arbitrary.
\begin{itemize}
	\item[(a)] Show that $U$ is a normal subgroup of $G$ and that $G / U$ is isomorphic to $D$.
	\item[(b)] True or False (with justification): $G \cong U \times D$
\end{itemize}
\soln
Let's look at the structure of $ U $. We have $ \matr{1}{x}{0}{1}\matr{1}{y}{0}{1}= \matr{1}{x+y}{0}{1} $. So, $ U $ is Abelian.

\begin{itemize}
	\item[(a)] Let $ g=\matr{a}{b}{0}{d} \in G$ and $ u=\matr{1}{x}{0}{1} \in U $. $ gug^{-1}=\matr{1}{\dfrac{ax}{d}}{0}{1} \in U$. So, $ U\unlhd G $.\\
	Let $ \phi:G\to D $ be a map $ \matr{a}{b}{0}{d}\mapsto \matr{a}{0}{0}{d} $.\\
	$ \matr{a_1}{b_1}{0}{d_1}\matr{a_2}{b_2}{0}{d_2}= \matr{a_1a_2}{a_1b_2 + b_1d_2}{0}{d_1d_2}\mapsto \matr{a_1a_2}{0}{0}{d_1d_2} $ is a homomorphism with kernel $ U $ and image $ D $.
	\item[(b)] $ G $ is nonabelian but the RHS is Abelian.\\
	$ \matr{1}{1}{0}{1}\matr{1}{1}{0}{2}=\matr{1}{3}{0}{2} \ne \matr{1}{1}{0}{2}\matr{1}{1}{0}{1}= \matr{1}{2}{0}{2} $.
\end{itemize}
\qed\\
\textbf{G7:} Let $G$ be a group and let $Z$ denote the center of $G$.
\begin{itemize}
	\item[(a)] Show that $Z$ is a normal subgroup of $G$.
	\item[(b)] Show that if $G / Z$ is cyclic, then $G$ must be abellan.
	\item[(c)] Let $D_6$ be the dihedral group of order 6 . Find the center of $D_6$.
\end{itemize}
\soln Let $ G $ be a group with center $ Z $.\\
\begin{itemize}
	\item[(a)] $ gzg^{-1}= zgg^{-1}=z\in Z $.
	\item[(b)] Let $ G/Z = C= <a>  $.\\
	 Let $ g_1,g_2\in G $. $ g_i Z= a^{k_i}Z \implies g_i = a^{k_i} z_i' z_i^{-1}  $. $ g_1g_2 = g_2g_1 = a^{k_1+k_2}z_1z_2z_1'z_2' $.
	 \item[(c)] $ D_6 = \{e,r,r^2,s,sr,sr^2\} $, $ rs=sr^2\ne sr $,$ r^2\cdot rs= s$, $ rs\cdot r^2 = ssrsr^2=sr^4=sr$. So, $ Z=\{e\} $.
\end{itemize}
\qed\\

\textbf{G8:} List all abelian groups of order 8 up to isomorphism. Identify which group on your list is isomorphic to each of the following groups of order 8 . Justify your answer.
\begin{itemize}
	\item[(a)] $(Z / 15 Z)^*=$ the group of units of the ring $Z / 15 Z$.
	\item[(b)] The roots of the equation $z^8-1=0 \text{ in } \mathbb{C}$.
	\item[(c)] $\mathbb{F}_8^{+}=$the additive group of the field $\mathbb{F}_8$ with eight elements.
\end{itemize}
\soln We use structure theorem for finitely generated Abelian group. $ G $ is isomorphic to one of these three groups. $ \Z_8, \Z_2\times\Z_4 $ or $ \Z_2\times \Z_2\times \Z_2 $.\\
\begin{itemize}
	\item[(a)]  $ (\Z/ 15\Z)^\times = (\Z/3\Z)^\times \times (\Z/5\Z)^\times = \Z_2\times\Z_4 $.
	\item[(b)] $ \mu_8 = e^{\dfrac{2\pi i}{8}} $ has order 8. So, it's isomorphic to $ \Z/8\Z $.
	\item[(c)] The field is of char $ 2 $. So, each element has order $ 2 $. So, it's isomorphic to $ \Z_2\times \Z_2\times \Z_2 $.
\end{itemize}
\qed\\
\textbf{G9:} Let $S_9$ denote the symmetric group on 9 elements.
\begin{itemize}
	\item[(a)] Find an element of $S_9$ of order 20.
	\item[(b)] Show that there is no element of $S_9$ of order 18 .
\end{itemize}
\soln Order of an element is the l.c.m. of the cycle lengths. 
\begin{itemize}
	\item[(a)] $ (12345)(6789) $.
	\item[(b)] We can't partition 9 into parts such that the lcm is $ 20 $.
\end{itemize}
\qed\\

\textbf{G10:} $G=\left\{\left[\begin{array}{cc}a & b \\ 0 & a^{-1}\end{array}\right]: a, b \in \mathbb{R}, a>0\right\}$ and $N=\left\{\left[\begin{array}{ll}1 & c \\ 0 & 1\end{array}\right]: c \in \mathbb{R}\right\}$ are groups under matrix multiplication.
\begin{itemize}
	\item[(a)] Show that $N$ is a normal subgroup of $G$ and that $G / N$ is isomorphic to the multiplicative group of positive real numbers $\mathbb{R}^{+}$.
	\item[(b)] Find a group $N^{\prime}$ with $N \subseteq N^{\prime} \subseteq G$, with both inclusions proper, or prove that no such $N^{\prime}$ exists.
\end{itemize}
\soln 
\begin{itemize}
	\item[(a)] Let $ \phi: G\to \R^+ $ be the homomorphism $ \matr{a}{b}{0}{a^{-1}}\mapsto a $(like solution 1.6) with kernel $ N $ and image $ \R^+ $. So, $ G/N\cong \R^+ $.  
	\item[(b)] Let $ <\dfrac{1}{2}>=\{\dfrac{1}{2^k}: k\in \Z\} $ be a subgroup of $ \R^+ $. The corresponding subgroup of $ G $ containing $ N $ is $ N'=\{\matr{\dfrac{1}{2^k}}{c}{0}{2^k}: k\in \Z, c\in\R\} $
\end{itemize}
\qed\\


\textbf{G11.1:}Let $R$ be a commutative ring with identity, and let $H$ be a subgroup of the group of units $R^{*}$ of $R$. Let $N=\{A \in \mathrm{GL}(n, R): \operatorname{det} A \in H\}$. Prove that $N$ is a normal subgroup of $\operatorname{GL}(n, R)$ and $\operatorname{GL}(n, R) / N \cong R^{*} / H$.\\
\textbf{G11.2:} Let $G$ be a group of order $2 p$ where $p$ is an odd prime. If $G$ has a normal subgroup of order 2 , show that $G$ is cyclic.\\
\soln

\begin{itemize}
	\item[1.] Consider the homomorphism\\
	$ \phi: GL(n,R)\to R^*/H $ by the map $ A\mapsto det(A) \pmod{H} $\\
	(Since $ R $ is commutative $ H $ is normal in $ R^* $). $ Ker(\phi) = N $ normal with full image(diagonal with a entry $ r $ and rest $ 1 $). So, we have the isomorphism.
	\item[2.] If $ G $ is abelian. $ G $ has element of order $ p $ and $ 2 $(Cauchy). Product of them has order $ lcm(2,p)=2p $. So, it generates $ G $.\\
	Let $ N=\{e,n\} $ where $ n^2=e $. $ gng^{-1}=n $($gng^{-1} =e \implies n=e$) i.e. $ n\in Z(G) $. So, $ G/Z(G) $ is either $ \Z_p $ or $ \Z_1 $ cyclic. So, $ G $ is Abelian.  
\end{itemize}
\qed\\
\textbf{G12:} Prove that every finitely generated subgroup of the additive group of rational numbers is cyclic.\\
\soln
Let $ G =<\dfrac{a}{b},\dfrac{c}{d}>$. Claim : $ G= <\dfrac{gcd(ad,bc)}{bd}> $. $ \dfrac{a}{b}= \dfrac{ad}{gcd(ad,bc)} \dfrac{gcd(ad,bc)}{bd}$ and $ \dfrac{c}{d}= \dfrac{bc}{gcd(ad,bc)} \dfrac{gcd(ad,bc)}{bd}$. On the other hand, by Bezout's identity $u\dfrac{a}{b}+v\dfrac{c}{d}= \dfrac{gcd(ad,bc)}{bd}  $. Now, use induction.

\qed\\

\textbf{G13:} Prove that any finite group of order $n$ is isomorphic to a subgroup of the orthogonal group $\mathrm{O}(n, \mathbb{R})$.\\

\soln(from stackexchange)\\
$S_{n}$ acts on $\mathbb{R}^n$ by the equation $$ \sigma . e_i= e_{\sigma(i)},$$ 
where $\lbrace e_i \vert i= 1,2,...,n\rbrace $ is the standard basis of $\mathbb{R}^n$ and $\sigma \in S_n$. Therefore we have a group morphism $$\varphi : S_n \rightarrow GL_n(\mathbb{R})$$ defined by $\varphi(\sigma)(e_i)= e_{\sigma(i)}.$
It is easy to check that $\varphi$ is one-one. Note that $\varphi(S_n) \subset \mathbb{O}(n)$, for $\langle \varphi(\sigma)(e_i), \varphi(\sigma)(e_j)\rangle~= ~\langle e_i, e_j\rangle$.\\
Now any finite group is a subgroup of $ S_n $.
\qed\\

\textbf{G14:} Prove that the group $\operatorname{GL}(2, \mathbb{R})$ has cyclic subgroups of all orders $n \in \mathbb{N}$. (Hint: The set of matrices $\left[\begin{array}{cc}a & b \\ -b & a\end{array}\right]$ where $a$ and $b$ are arbitrary real numbers, is a subring of the ring of $2 \times 2$ matrices which is isomorphic to $\mathbb{C}$.)\\
\soln
Use the hint. We have a cyclic subgroup of order $ n $ generated by the $ n $-th root of unity $ \mu_n  $ in $ \C $. Take it's image in $ GL(2,\R) $.

\qed\\

\textbf{G15:} Let $H_{1}$ be the subgroup of $\mathbb{Z}^{2}$ generated by $\{(1,3),(1,7)\}$ and let $H_{2}$ be the subgroup of $\mathbb{Z}^{2}$ generated by $\{(2,4),(2,6)\}$. Are the quotient groups $G_{1}=\mathbb{Z}^{2} / H_{1}$ and $G_{2}=\mathbb{Z}^{2} / H_{2}$ isomorphic?\\
\soln
$ H_1=<(1,3),(1,7)>=<(1,3),(0,4)>=<(1,-1),(0,4)> $. $ \Z^2/H_1=\Z_4 $ with the generator $ (0,1)+H_1 $ of order $ 4 $ (easy to show order divides $ 4 $, but order isn't $ 2 $).\\
$ H_2=<(2,0),(0,2)> \Z^2/H_2=\Z_2\times \Z_2$. Not isomorphic by comparing the order.
\qed\\

\textbf{G16:} Let $H$ and $N$ be subgroups of a group $G$ with $N$ normal. Prove that $H N=N H$ and that this set is a subgroup of $G$.\\

\soln
The first proof is trivial by definition of normal subgroup: $ hN=Nh $.\\
$ n_1h_1n_2h_2=n_1n'_2h'_1h_2=n_3h_3\in NH $.\\
$ (nh)^{-1}=h^{-1}n^{-1}\in HN=NH $.
\qed\\

\textbf{G17:} Let $G=\mathbb{Z} / 2 \mathbb{Z} \oplus \mathbb{Z} / 6 \mathbb{Z} \oplus \mathbb{Z} / 30 \mathbb{Z}$ and let $H=\mathbb{Z} / 4 \mathbb{Z} \oplus \mathbb{Z} / 20 \mathbb{Z}$. Express the abelian group Hom $(G, H)$ of homomorphisms from $G$ to $H$ as a direct sum of cyclic groups.\\
\soln
We use the fact 
$$ Hom(\Z_n,\Z_m)=\{\texttt{element of }\Z_m \texttt{ with order dividing } n\}=\Z_(gcd(n,m)) $$
$ Hom(G,H)=Hom(\Z_2,H)\oplus Hom(\Z_6,H)\oplus Hom(\Z_{30},H)=\Z_2\oplus \Z_2 \oplus \Z_2\oplus \Z_2 \oplus \Z_2 \oplus \Z_{10} $.

\qed\\

\textbf{G18:} Let $G$ be an abelian group generated by $x, y, z$ subject to the relations

$$
\begin{aligned}
	15 x+3 y & =0 \\
	3 x+7 y+4 z & =0 \\
	18 x+14 y+8 z & =0
\end{aligned}
$$

\begin{itemize}
	\item[(a)] Write $G$ as a product of two cyclic groups.\\
	\item[(b)] Write $G$ as a direct product of cyclic groups of prime power order.\\
	\item[(c)] How many elements of $G$ have order 2?
\end{itemize}

\soln
We need to calculate the Smith Normal form of the matrix(row/column swap, $ R_i\to R_i+ kR_j $, $ C_i\to C_i+ kC_j $, multiply by $ -1 $)
$ \begin{pmatrix}
	15 & 3 & 0\\
	3 & 7 & 4\\
	18 & 14 & 8
\end{pmatrix} $.\\
$ \begin{pmatrix}
	15 & 3 & 0\\
	3 & 7 & 4\\
	12 & 0 & 0
\end{pmatrix} \to
\begin{pmatrix}
	3 & 3 & 0\\
	3 & 7& 4\\
	12 & 0 & 0
\end{pmatrix} \to
\begin{pmatrix}
	3 & 3 &0 \\
	0 & 4 & 4 \\
	12 & 0 &0 
\end{pmatrix}\to
\begin{pmatrix}
	3 & 0 &0 \\
	0 & 4 & 4 \\
	12 & -12 &0 
\end{pmatrix}\to
\begin{pmatrix}
	3 & 0 &0 \\
	0 & 4 & 4 \\
	0 & -12 &0 
\end{pmatrix}\to
\begin{pmatrix}
	3 & 4 &4 \\
	0 & 4 & 4 \\
	0 & -12 &0 
\end{pmatrix}
$\\
$ \begin{pmatrix}
	3 & 1 & 1\\
	0 & 4 & 4\\
	0 & -12 & 0
\end{pmatrix} \to
\begin{pmatrix}
	1 & 1 & 3\\
	4 & 4& 0\\
	0 & -12 & 0
\end{pmatrix} \to
\begin{pmatrix}
	1 & 0 &0 \\
	4 & 0 & -12 \\
	0 & -12 &0 
\end{pmatrix}\to
\begin{pmatrix}
	1 & 0 &0 \\
	0 & 0 & -12 \\
	0 & -12 &0 
\end{pmatrix}\to
\begin{pmatrix}
	1 & 0 &0 \\
	0 & 12 & 0 \\
	0 & 0 &12 
\end{pmatrix}
$
\begin{itemize}
	\item[(a)] So, $ G= \Z_{12}\oplus\Z_{12} $.
	\item[(b)] $ G= \Z_3\times \Z_3\times \Z_4\times\Z_4 $
	\item[(c)] 3 order 2 element: $ (6,0),(0,6),(6,6)\in C_{12}\times C_{12} $
\end{itemize}
\qed\\

\textbf{G19:} Let $\mathbb{F}$ be a field and let

$$
H(\mathbb{F})=\left\{\left[\begin{array}{ccc}
	1 & a & b \\
	0 & 1 & c \\
	0 & 0 & 1
\end{array}\right]: a, b, c \in \mathbb{F}\right\}
$$

\begin{itemize}
	\item[(a)] Verify that $H(\mathbb{F})$ is a nonabelian subgroup of $\mathrm{GL}(3, \mathbb{F})$.
	\item[(b)] If $|\mathbb{F}|=q$, what is $|H(\mathbb{F})|$ ?
	\item[(c)] Find the order of all elements of $H(\mathbb{Z} / 2 \mathbb{Z})$.
	\item[(d)] Verify that $H(\mathbb{Z} / 2 \mathbb{Z}) \cong D_{8}$, the dihedral group of order 8 .
\end{itemize}
\soln
\begin{itemize}
	\item[(a)] $ \begin{pmatrix}
		1 & a_1 & b_1\\
		0 & 1 & c_1\\
		0 & 0 & 1
	\end{pmatrix} 
\begin{pmatrix}
	1 & a_2 & b_2\\
	0 & 1 & c_2\\
	0 & 0 & 1
\end{pmatrix} = 
\begin{pmatrix}
	1 & a_1+a_2 & b_1+b_2 +a_1c_2\\
	0 & 1 & c_1+c_2\\
	0 & 0 & 1
\end{pmatrix}$.\\
Inverse of $ \begin{pmatrix}
	1 & a_1 & b_1\\
	0 & 1 & c_1\\
	0 & 0 & 1
\end{pmatrix} $ is $ \begin{pmatrix}
1 & -a_1 & a_1c_1-b_1\\
0 & 1 & -c_1\\
0 & 0 & 1
\end{pmatrix} $. Non Abelian for the (1,3)th entry.
\item[(b)] We have $ q $ choices for each of $ a,b $ and $ c $. So $ q^3 $.
\item[(c,d)] $ e=I_3 , r= \begin{pmatrix}
	1 & 1 & 1\\
	0 & 1 & 1\\
	0 & 0 & 1
\end{pmatrix}, s= \begin{pmatrix}
1 & 1 & 0\\
0 & 1 & 0\\
0 & 0 & 1
\end{pmatrix}$.
\end{itemize}
\qed\\

\textbf{G20:} Let $R$ be an integral domain and let $G$ be a finite subgroup of $R^{*}$, the group of units of $R$. Prove that $G$ is cyclic.\\
\soln
Note that this result is true for Fields. Any subgroup of units of integral domain is a subgroup of it's quotient field's units. Thus the result follows. In general it follows from the roots of the polynomial $ x^n-1 $ in Field or integral domain(at most $ n $ many roots).
\qed\\

G21. Let $\alpha$ and $\beta$ be conjugate elements of the symmetric group $S_{n}$. Suppose that $\alpha$ fixes at least two symbols. Prove that $\alpha$ and $\beta$ are conjugate via an element $\gamma$ of the alternating group $A_{n}$.\\

\soln
\qed\\

G22. Are $(13)(25)$ and $(12)(45)$ conjugate in $S_{5}$ ? If you say "yes", find an element giving the conjugation; if you say "no", prove your answer.\\

\soln
\qed\\

G23. (a) Suppose that $G$ is a group and $a, b \in G$ are elements such that the order of $a$ is $m$ and the order of $b$ is $n$. If $a b=b a$ and if $m$ and $n$ are relatively prime, show that the order of $a b$ is $m n$.\\
(b) Prove that an abelian group of order $p q$, where $p$ and $q$ are distinct primes, must be cyclic.\\
(c) If $m$ and $n$ are relatively prime, must a group of order $m n$ be cyclic? Justify your answer.

G24. Let $\varphi: G \rightarrow H$ be a surjective group homomorphism and let $N$ be a normal subgroup of $G$. Show that $\varphi(N)$ is a normal subgroup of $H$. What happens if $\varphi$ is not surjective? Explain your answer.\\
G25. Let $Q=\{1,-1, i,-i, j,-j, k,-k\}$ be the quaternion group and $N=\{1,-1, i,-i\}$. Show that $N$ is a normal subgroup of $Q$. Describe the quotient group $Q / N$.\\
G26. Let $G$ be a finite abelian group of odd order. If $\varphi: G \rightarrow G$ is defined by $\varphi(a)=a^{2}$ for all $a \in G$, show that $\varphi$ is an isomorphism. Generalize this result.\\
G27. Prove that the direct product of two infinite cyclic groups is not cyclic.\\
G28. Prove that if a group has exactly one element of order two, then that element is in the center of the group.\\
G29. Prove that a group of order 30 can have at most 7 subgroups of order 5 .

G30. Let $H=\{1,-1, i,-i\}$ be the subgroup of the multiplicative group $G=\mathbb{C}^{*}=\mathbb{C} \backslash\{0\}$ consisting of the fourth roots of unity. Describe the cosets of $H$ in $G$, and show that the quotient $G / H$ is isomorphic to $G$.\\
G31. (a) Show that the set of all elements of finite order in an abelian group form a subgroup.\\
(b) Let $G=\mathbb{R} / \mathbb{Z}$. Show that the set of elements of $G$ of finite order is the subgroup $\mathbb{Q} / \mathbb{Z}$.

G32. Determine all the finite groups which have exactly 3 conjugacy classes.\\
G33. (a) Find a Sylow 2-subgroup of $S_{4}$ and identify its isomorphism type.\\
(b) How many Sylow 2-subgroups of $S_{4}$ are there? Please justify your answer.

G34. (a) Determine the number of Sylow $p$-subgroups of $S_{5}$ for each prime factor $p$ of $\left|S_{5}\right|$. Please prove your assertion.\\
(b) Identify the isomorphism type of each Sylow $p$-subgroup of $S_{5}$. Please prove your assertion.

G35. Let $p<q$ be prime numbers such that $p \mid(q-1)$. Show there exists a unique nonabelian group of order $p q$ up to isomorphism.\\
G36. (a) Define what it means for a group $G$ to act on a set $A$.\\
(b) The group $\mathrm{GL}_{2}(\mathbb{C})$ acts by left-multiplication on the set of matrices $M_{2,5}(\mathbb{C})$. Describe the orbits. How many are there?