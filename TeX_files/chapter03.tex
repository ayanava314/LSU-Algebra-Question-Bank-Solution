\chapter{Module Theory}
\textbf{M1:}Let $\mathbb{Z}\left[\frac{1}{2}\right]$ denote the subring of $\mathbb{Q}$ generated by $\mathbb{Z}$ and $\frac{1}{2}$. Is $\mathbb{Z}\left[\frac{1}{2}\right]$ finitely generated as a $\mathbb{Z}$-module? Justify your answer.\\
M2. Let $\mathbb{Z}\left[\frac{1}{2}\right]$ denote the subring of $\mathbb{Q}$ generated by $\mathbb{Z}$ and $\frac{1}{2}$. Prove or disprove: $\mathbb{Z}\left[\frac{1}{2}\right]$ is a free $\mathbb{Z}$-module.\\
M3. (a) Show that $\mathbb{Q}$ is a torsion-free $\mathbb{Z}$-module.\\
(b) Is $\mathbb{Q}$ a free $\mathbb{Z}$-module? Justify your answer.

M4. Show that $\mathbb{Z} / 3 \mathbb{Z}$ is a $\mathbb{Z} / 6 \mathbb{Z}$-module and conclude that it is not a free $\mathbb{Z} / 6 \mathbb{Z}$-module.\\
M5. Let $N$ be a submodule of an $R$-module $M$. Show that if $N$ and $M / N$ are finitely generated, then $M$ is finitely generated.\\
M6. Let $G$ be the abelian group with generators $x, y$, and $z$ subject to the relations

$$
\begin{array}{r}
	5 x+9 y+5 z=0 \\
	2 x+4 y+2 z=0 \\
	x+y-3 z=0 .
\end{array}
$$

Determine the elementary divisors of $G$ and write $G$ as a direct sum of cyclic groups.\\
M7. Let $R$ be a ring and let $f: M \rightarrow N$ be a surjective homomorphism of $R$-modules, where $N$ is a free $R$-module. Show that there exists an $R$-module homomorphism $g: N \rightarrow M$ such that $f \circ g=1_{N}$. Show that $M=\operatorname{Ker}(f) \oplus \operatorname{Im}(g)$.\\
M8. Let $R$ be an integral domain and let $M$ be an $R$-module. A property $P$ of $M$ is said to be hereditary if, whenever $M$ has property $P$, then so does every submodule $N$ of $M$. Which of the following properties of $M$ are hereditary? If a property is hereditary, give a brief reason. If it is not hereditary, give a counterexample.\\
(a) Free\\
(b) Torsion\\
(c) Finitely generated

M9. Let $R$ be an integral domain. Determine if each of the following statements about $R$-modules is true or false. Give a proof or counterexample, as appropriate.\\
(a) A submodule of a free module is free.\\
(b) A submodule of a free module is torsion-free.\\
(c) A submodule of a cyclic module is cyclic.\\
(d) A quotient module of a cyclic module is cyclic.

M10. Let $M$ be an $R$-module and let $f: M \rightarrow M$ be an $R$-module endomorphism which is idempotent, that is, $f \circ f=f$. Prove that $M \cong \operatorname{Ker}(f) \oplus \operatorname{Im}(f)$.\\
M11. Prove that $\operatorname{Hom}_{\mathbb{Z}}(\mathbb{Z} / n \mathbb{Z}, \mathbb{Z} / m \mathbb{Z}) \cong \mathbb{Z} / d \mathbb{Z}$, where $d$ is the greatest common divisor of $n$ and $m$.\\
M12. Compute $\operatorname{Hom}_{\mathbb{Z}}(\mathbb{Z}, \mathbb{Q})$ and $\operatorname{Hom}_{\mathbb{Z}}(\mathbb{Q}, \mathbb{Z})$.\\
M13. Let $R$ be a commutative ring with 1 and let $I$ and $J$ be ideals of $R$. Prove that $R / I \cong R / J$ as $R$-modules if and only if $I=J$. Suppose we only ask that $R / I$ and $R / J$ be isomorphic as rings. Is the same conclusion valid? (Hint: Consider $F[X] /\langle X-a\rangle$ for $a \in F$.)\\
M14. Let $M \subseteq \mathbb{Z}^{n}$ be a $\mathbb{Z}$-submodule of rank $n$. Prove that $\mathbb{Z}^{n} / M$ is a finite group.\\
M15. Let $G, H$, and $K$ be finite abelian groups. If $G \times K \cong H \times K$, then prove that $G \cong H$.\\
M16. Let $G$ be an abelian group and $K$ a subgroup. For each of the following statements, decide if it is true or false. Give a proof or provide a counterexample, as appropriate.\\
(a) If $G / K \cong \mathbb{Z}^{2}$, then $G \cong K \oplus \mathbb{Z}^{2}$.\\
(b) If $G / K \cong \mathbb{Z} / 2 \mathbb{Z}$, then $G \cong K \oplus \mathbb{Z} / 2 \mathbb{Z}$.

M17. Let $F$ be a field and let $V$ and $W$ be vector spaces over $F$. Make $V$ and $W$ into $F[X]$-modules via linear operators $T$ on $V$ and $S$ on $W$ by defining $X \cdot v=T(v)$ for all $v \in V$ and $X \cdot w=S(w)$ for all $w \in W$. Denote the resulting $F[X]$-modules by $V_{T}$ and $W_{S}$ respectively.\\
(a) Show that an $F[X]$-module homomorphism from $V_{T}$ to $W_{S}$ consists of an $F$-linear transformation $R: V \rightarrow W$ such that $R T=S R$.\\
(b) Show that $V_{T} \cong W_{S}$ as $F[X]$-modules if and only if there is an $F$-linear isomorphism $P: V \rightarrow W$ such that $T=P^{-1} S P$.\\
M18. Let $G=\mathbb{Z} / 4 \mathbb{Z} \oplus \mathbb{Z} / 6 \mathbb{Z} \oplus \mathbb{Z} / 9 \mathbb{Z} \oplus \mathbb{Z} / 10 \mathbb{Z}$. Determine the elementary divisors and invariant factors of $G$.\\
M19. (a) Find a basis and the invariant factors of the submodule $N$ of $\mathbb{Z}^{2}$ generated by $x=(-6,2)$, $y=(2,-2)$ and $z=(10,6)$.\\
(b) From your answer to part (a), what is the structure of $\mathbb{Z}^{2} / N$ ?

M20. Let $R$ be a ring and let $M$ be a free $R$ module of finite rank. Prove or disprove each of the following statements.\\
(a) Every set of generators contains a basis.\\
(b) Every linearly independent set can be extended to a basis.

M21. Let $R$ be a ring. An $R$-module $N$ is called simple if it is not the zero module and if it has no submodules except $N$ and the zero submodule.\\
(a) Prove that any simple module $N$ is isomorphic to $R / M$, where $M$ is a maximal ideal.\\
(b) Prove Schur's Lemma: Let $\varphi: S \rightarrow S^{\prime}$ be a homomorphism of simple modules. Then either $\varphi$ is zero, or it is an isomorphism.\\
M22. (a) Give an example of a prime ideal in a ring that is not maximal.\\
(b) Describe $\operatorname{Spec}(\mathbb{C}[x])$ (polynomial ring in one variable over the complex numbers).\\
(c) Describe $\operatorname{Spec}(\mathbb{R}[x])$.